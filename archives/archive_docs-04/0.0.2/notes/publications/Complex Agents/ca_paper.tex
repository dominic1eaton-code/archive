\documentclass[a4paper,12pt]{book}
\usepackage[T1]{fontenc}
\usepackage[utf8]{inputenc}
\usepackage{titlesec}
 
\newtheorem{definition}{Definition}[section]
\newtheorem{theorem}{Theorem}[section]
\newtheorem{corollary}{Corollary}[theorem]
\newtheorem{lemma}[theorem]{Lemma}

\renewcommand{\rmdefault}{ppl}
\renewcommand{\bfdefault}{b}
\usepackage[scaled]{helvet}
\renewcommand{\ttdefault}{pcr}
\linespread{1.05}

% tgcursor helvet tgbonum lmodern
\normalfont % in case the EC fonts aren't available

\usepackage{blindtext}
\usepackage{mathrsfs,amsmath}
\usepackage[utf8]{inputenc}
\usepackage{titlesec}
\usepackage{amsmath}
\usepackage{algorithm}
\usepackage[noend]{algpseudocode}
\usepackage[toc,page]{appendix}
\usepackage{graphicx}
\usepackage{lipsum}
%\documentclass{amsart}

\setcounter{secnumdepth}{4}

\titleformat{\paragraph}
{\normalfont\normalsize\bfseries}{\theparagraph}{1em}{}
\titlespacing*{\paragraph}
{0pt}{3.25ex plus 1ex minus .2ex}{1.5ex plus .2ex}




\title{Toward Complex Mobility Agent \\ Behavior Control \\
\noindent\rule{10cm}{0.4pt}}
\author{Dominic Eaton}


\begin{document}  
\maketitle

\pagebreak


\tableofcontents{}

\pagebreak

\part{Preliminaries}
    \chapter{The Question}
        \begin{center}
            \textit{How could 
            one control a complex system of autonomous mobility agents ?}
            \linebreak 
            \linebreak 
            \textit{What are the "right" conditions to give rise to such control mechanisms ?}
        \end{center}
        
        Shall the entirety of this work be dedicated to answering these questions and understanding their meaning. Naturally is it necessary to first find a suitable definition for a  \textit{complex system}, ideally in terms of its composition and how such a system can be subject to change (over time and space), and furthermore may the there be a desire to find appropriate, concise, and exact representation of such modalities. Once these fundamental and formal definitions, description  and principles have been established, can one then hope to find the means for designing and implementing control mechanisms of such systems, where the aim of these control systems are to provide functionalities of agents, or essentially make the agents perform some service that may provide utility to some end user. 
                
        
    \chapter{Introduction}
        We begin with a definition: \\
        
        \begin{definition}
            Complex Autonomous System \\
            A complex autonomous system capable of (controlled) manipulation as defined by the 6-tuple :
            \begin{center}
                $ \mathscr{C} = \{S, C, E, A, \mathcal{M}, \Theta \}$
            \end{center}
        \end{definition}
 
        \section{The Progression}
        
    \chapter{Chaos and Order}
    
    \chapter{Energy}
    
    \chapter{Space and Time}
    
    \chapter{Complexity}

    \chapter{Modalities}
    
\part{System}
    \chapter{Modes of Composition}
        How the agent[s] comes together
        
    \chapter{Modes of Change}
        How the agent[s] change: originated by composition
            
    \chapter{Modes of Interaction}
        How the agents interact: originated by change
       
    \chapter{Modes of Behavior}
        How the agents behave: originated by interaction
        
    \chapter{Modes of State and Evolution}
        How the agents form states (configuration): originated by behavior (and other previous modes)
        
\part{Control}   
    \chapter{Modes of Observation}
    
    \chapter{Modes of Control}
    
% APPENDICES
\begin{appendices}
    \chapter{Notations} 
    
    \chapter{Definitions}
    
    
\end{appendices}

% BIBLIOGRAPHY 
\begin{thebibliography}{9}
    \bibitem{latexcompanion} 
        Michel Goossens, Frank Mittelbach, and Alexander Samarin. 
        \textit{The \LaTeX\ Companion}. 
        Addison-Wesley, Reading, Massachusetts, 1993.
     
    \bibitem{einstein} 
        Albert Einstein. 
        \textit{Zur Elektrodynamik bewegter K{\"o}rper}. (German) 
        [\textit{On the electrodynamics of moving bodies}]. 
        Annalen der Physik, 322(10):891–921, 1905.
     
    \bibitem{knuthwebsite} 
        Knuth: Computers and Typesetting,
        \\\texttt{http://www-cs-faculty.stanford.edu/\~{}uno/abcde.html}
\end{thebibliography}

\end{document}
    