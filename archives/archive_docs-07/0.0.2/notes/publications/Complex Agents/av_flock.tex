\documentclass[a4paper,12pt]{book}
\usepackage[T1]{fontenc}
\usepackage[utf8]{inputenc}
\usepackage{titlesec}
 
\newtheorem{definition}{Definition}[section]
\newtheorem{theorem}{Theorem}[section]
\newtheorem{corollary}{Corollary}[theorem]
\newtheorem{lemma}[theorem]{Lemma}

\renewcommand{\rmdefault}{ppl}
\renewcommand{\bfdefault}{b}
\usepackage[scaled]{helvet}
\renewcommand{\ttdefault}{pcr}
\linespread{1.05}
\normalfont % in case the EC fonts aren't available

\usepackage{blindtext}
\usepackage{mathrsfs,amsmath}
\usepackage[utf8]{inputenc}
\usepackage{titlesec}
\usepackage{amsmath}
\usepackage{algorithm}
\usepackage[noend]{algpseudocode}
\usepackage[toc,page]{appendix}
\usepackage{graphicx}
\usepackage{lipsum}
%\documentclass{amsart}

\setcounter{secnumdepth}{4}

\titleformat{\paragraph}
{\normalfont\normalsize\bfseries}{\theparagraph}{1em}{}
\titlespacing*{\paragraph}
{0pt}{3.25ex plus 1ex minus .2ex}{1.5ex plus .2ex}




\title{Toward the Simulation of Vehicular Flocks (Crowds) 
\noindent\rule{10cm}{0.4pt}}
\author{Dominic Eaton}
\date

\begin{document}
    \maketitle  
    
    \tableofcontents
    
    \part{Preliminaries}
    
        \chapter{Prelude}
        \section{A side thought}
            The work completed by practitioners of Thermodynamics and related derivative fields have caused wonder as to the role of which entropy and energy have to play in relation to the complex autonomous system and its capability to produce emergent behaviors. Other principles such as hierarchy to structure complexity, ideal morphology as a means for optimal functionality, and …, are also of consideration, and one must wonder at how these elements of thermodynamics, statistical mechanics, information theory, formal theories of language and computation, economics, game theory, distributed systems, optimal control, security, privacy and trust notions, governance and finance, computer networking, machine learning, artificial intelligence, numerical optimization, sociology, and biology (and many other fields); how must all of these vast subjects come together in smooth and elegantly simple cohesion to produce systems of great complexity that can solve problems of equal or greater complexity. My time spent wondering, wandering, searching, and slowly compiling and piecing together information from these fields (for which significantly more progress is still to be made) has perhaps made me realize that starting at the most fundamental of levels to form theories may provide the best way as to build solutions and find answers to the questions for which are desired to be sought. Does the work here merely desire to create a new class of autonomous mobility transportation systems for enhanced mobility, but greater than this is there a lingering curiosity as the point of connection and intersection among all of the fields previously mentioned, and if in fact these fields truly do connect at some central seed from which sprouts the roots of complex system theory (along with the theory of agents). Is there a belief that a system reaches a point of "evolution" when it comes to the point of gaining enough "data / information" that it can pull all of these things together to form something new moving forward (here considered to be a process of maturation). This threshold / singularity point (though I am certainly not saying it has be reached yet [or if shall ever be reached (any time soon that is)]) is such that understanding can be born from the newly created foundations, and do I believe this work to not be of exception to this (surmised) principle. This is all to say that even if the initial ideas, premises, or concepts are less than "ideal" (meaning that they are erroneous in some way), can I not help but feel that, in due time, the mere asking of the question shall be, in essence, a sufficient mechanism as to generate the proper solutions and reasons as to see to he realization of the final goals of this work. A process of germination which entails planting a seed of the idea begins here, with the hope that it shall one day grow and develop into a mature idea enhancing (or redefining) what it means to have a truly "efficient" and "smart" transportation system, is what can be hoped for at least here by continuing the development of this work. Should it be noted that I need not be the sole propriety of the ideas attempting to be promoted by this work, but do these ideas exist with independence, on their own right, for which they are free for the development and influence of all for whom are willing read them. Was there once a time where one could have believed that creation was owned exclusively by a creator, but now is there the wonder if creation lives only vicariously through a given creator, and if it is in actuality the passing through several vehicles of "creator" that creation comes to exist and mature.
        
        \section{Objectives}
        Vehicular Crowding Strategies Objectives:
            \begin{enumerate}
            \item Enhancing V2V communication networks which are sensitive to the nature volatility incurred by (high) mobility systems (fast moving vehicles)
            \item Creating traffic systems with flexible morphology properties in order to better navigate (highly) constrained roadway systems / environments
            \item Creating sustainable movement strategies for vehicular systems that allow them to more efficiently utilize their resources while on roadways (along with creation of "feed back loop" systems using [autonomous] mobility systems for vehicular traffic)
            \item Minimizing the costs and inefficiencies generated by current vehicular traffic systems (energy wastage, non-usage of vehicles (the "parked car problem", vehicular collisions and accidents, traffic congestion, etc…)
            \item Enhance human mobility …
            \item Study emergent behaviors of a chaotic system …
            \end{enumerate}
        
        \chapter{Introduction}
            Is there no doubt to the imminence of the advent of another technological evolution, even in this age of technological advancement. Can one already hear the whispers and see the signs of something on the horizon, that has begun to mold its previously amorphous form into a shape that shall spell the end of old schools of thought, and the beginning of new ones. As with all of these early works (related to the previous works in this series on complex autonomous agents) is there merely the desire to determine a conceptual model of future models to come that shall perhaps, in time, become common place... \\ \\
            The Self Driving Car appears to be an ideal specimen of study, as its development sees the combination and culmination of many different fields, all of which strive at achieving a singular (and thankfully constrained) objective. Does work strive to use the example of the Self Driving Car as a means to bring light to deeper considerations on the topic of complex autonomous agents (of mobility), and methodologies for performing systematic and formal treatments of these systems. As such the question here is not necessarily when shall self driving cars be prevalent ? or even how can one construct a better self driving car ? but rather, when these vehicles finally enter into society (in any capacity) what shall their utility be ? how can we (humans, but can be extended to other organisms as well) use such a technology ? It is believed in this work that the true value of the self driving car shall certainly not come from having single or even few of these vehicles permeated our roadways, but when masses of these vehicles, possessing such abilities of higher level cognition and functionality, shall come together with the expectation that emergent properties shall come forth and the synergistic nature of this novel of complex autonomous networks shall produce a far more valuable tool that can accomplish far greater things than what may have been originally intended. So now looking at these "mass of self driving cars", what shall they become ? what possibilities exist with this system ? how can one characterize these systems and their properties ? how can one test or demonstrate the efficacy of such a system ? These questions are the beginning of a long line of inquiry into the nature of what (networks of) complex autonomous systems are and what they shall have the intention of being in times to come (where the self driving car is the most tangible archetypal example, for now). ... \\ \\
            Beyond the scope of study does one imagine what applications can become possible with the advent of such tools. In the event of disaster, will these tools be able to maintain themselves and their patrons (those who use the tools). In trying to reach places far and unknown (space or the deepest parts of the oceans) to what extent can these tools enhance our own human capability to achieve such (large [for lack of a better term]) goals ? Is there the belief that a systematic and formal treatment of complex systems (of autonomy) is necessary to provide, at the least, an incipient means of answering such questions with sufficient and sustainable solutions. Can one imagine the difficulties that may come when attempting to build complex theories and systems on a fragmented, unstable or incomplete foundation, and as such, it is the goal of this work to begin to ask questions that shall lead hopefully to a definition and formulation of a firm foundation for complex autonomous systems, from which truly sustainable and enduring solutions can be produced.
    
    \part{The Stack}
    
        \chapter{Mobility Layer}
            
        \chapter{Communications Layer}
        
        \chapter{Data and Information Layer}
            
        \chapter{Intelligence Layer}
        
        \chapter{Application Layer}
        
        \chapter{Management and Organization Layer}
            
    
\end{document}
