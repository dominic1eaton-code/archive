\documentclass[a4paper,12pt]{book}
\usepackage[T1]{fontenc}
\usepackage[utf8]{inputenc}
\usepackage{titlesec}
 
\renewcommand{\rmdefault}{ppl}
\renewcommand{\bfdefault}{b}
\usepackage[scaled]{helvet}
\renewcommand{\ttdefault}{pcr}
\linespread{1.05}
\normalfont % in case the EC fonts aren't available

\usepackage{blindtext}
\usepackage{mathrsfs,amsmath}
\usepackage[utf8]{inputenc}
\usepackage{titlesec}
\usepackage{amsmath}
\usepackage{algorithm}
\usepackage[noend]{algpseudocode}
\usepackage[toc,page]{appendix}
\usepackage{graphicx}
\usepackage{lipsum}
%\documentclass{amsart}

\setcounter{secnumdepth}{4}

\titleformat{\paragraph}
{\normalfont\normalsize\bfseries}{\theparagraph}{1em}{}
\titlespacing*{\paragraph}
{0pt}{3.25ex plus 1ex minus .2ex}{1.5ex plus .2ex}




\title{The Procedural Generation \\ of Complex Environments
\noindent\rule{10cm}{0.4pt}}
\author{Dominic Eaton}


\begin{document}  
\maketitle

\pagebreak


\tableofcontents{}

\pagebreak

\part{Preliminaries}

    \chapter{Introduction}
        \textit{"to simulate all of the possibilities... to create a dream that never ends... all so that to ensure that our beliefs never die."} \\
        
        It is no secret that the human mind is in a constant state of simulation. Always is it imagining what could be, what was, and what is. Naturally, I feel, must one begin to ponder that if such mechanism were to be replicated in some way, what possibilities or potentials could be unlocked. Do I not know that nature of how the human mind comes to do and create the fantastic things that it does, but I shall at least endeavor to mimic such process in autonomous mechanical agents, of sufficient complexity, who can be controlled for performing services. Can I only \textit{imagine} at how endowing such a servicing agent with the ability to simulate a world of endless possibilities shall put such agents well on their way to achieving some semblance of \textit{"true"} intelligence. Do I simply not feel that it is adequate for an agent to understand the \textit{real world} as it currently is, but do I also think that an agent must also have some understanding of the world *as it could be*. Do I feel that this capacity to imagine is what gives humans their edge of other lifeforms, and so such do I think that this is a necessary component for facilitating complex behavior of any significance. 
        
        \section{The Progression}
        

        
    \chapter{Coherent Noise}
        \section{Mathematical Formulation}
    
        \section{Perlin Noise}
        
        \section{Simplex Noise}
        
        \section{Fractal Noise}
        
        \section{Color Noise}
        
    \chapter{Tensors}
    
    \chapter{Spaces}

\part{Procedural World Generation}
    \chapter{The Pipeline}
        \section{Static Element Generation}
        
        \section{Dynamic Element Generation}
        
        \section{Refined Element Generation}
        
    \chapter{Universes}
        \section{Space and Time}
            \subsection{Spacetime}
            We begin our path toward procedural generation by defining the concepts of space and time. 
            
            \subsection{Reference Frames}
        
    \chapter{Worlds}
        \section{Solar Time}
            
        \section{Circadian Rhythm}   
        
        \section{Nychthemeron}
        
    \chapter{Environments}
        
        \section{Weather}
        
        \section{Climate}
            
        \section{Biomes}
            
    \chapter{Terrain}
        \section{Morphology}
            \subsection{Noise Generation}
            
            \subsection{Elevations}
            
            \subsection{Water}
            
            
            
        \section{Information Maps}
        
            \subsection{Height Map}

            \subsection{Population Map}
            
            \subsection{Weather Map}
            
            \subsection{Water Map}
            
        \section{Smart Codes}
    
    \chapter{Roads}
        
        \section{Stationing}
        
    \chapter{Settlements}
    
    \chapter{Buildings}
    
        \section{Parcels}
        
        \section{Footprints}
        
        \section{Shape Grammars}
    
    \chapter{Infrastructure}
    
    \chapter{Transit Systems}

\part{Procedural Agent Generation}

    \chapter{Agents}
    
    \chapter{Animals}
    
    \chapter{Pedestrians}
    
    \chapter{Vehicles}
    
        \section{Traffic}
        
        \section{Transit Networks}
    
    
\part{Procedural Character Generation}

    \chapter{Morphology}
    
    \chapter{Animation}
    
    \chapter{Material}
    

\part{Applications}

    \chapter{Distributed Systems}
    
    \chapter{Networking}
    
    \chapter{Artificial Intelligence}
    
    \chapter{Agent Planning and Control}
    
    \chapter{Complex Systems}

% APPENDICES
\begin{appendices}
    \chapter{Notations} 
    
    \chapter{Definitions}
    
    \chapter{Tensors}
        \section{Fundamentals}
            \subsection{Prelims}
            
            \subsection{Degenerate Points}
            
        \section{Fields}
            \subsection{Scalar Fields}
            
            \subsection{Vector Fields}
            
            \subsection{Tensor Fields}
        
        \section{Visualization}
        
            \section{Numerical Integration}
            
    \chapter{Computational Geometry}
        \section{Prelims}
        
        \section{Triangulation}
        
        \section{Convex Hulls}
        
        \section{Geometry Manipulation}
            \subsection{Partitioning}
            
            \subsection{Merging}
        
\end{appendices}

% BIBLIOGRAPHY 
\begin{thebibliography}{9}
    \bibitem{latexcompanion} 
        Michel Goossens, Frank Mittelbach, and Alexander Samarin. 
        \textit{The \LaTeX\ Companion}. 
        Addison-Wesley, Reading, Massachusetts, 1993.
     
    \bibitem{einstein} 
        Albert Einstein. 
        \textit{Zur Elektrodynamik bewegter K{\"o}rper}. (German) 
        [\textit{On the electrodynamics of moving bodies}]. 
        Annalen der Physik, 322(10):891–921, 1905.
     
    \bibitem{knuthwebsite} 
        Knuth: Computers and Typesetting,
        \\\texttt{http://www-cs-faculty.stanford.edu/\~{}uno/abcde.html}
\end{thebibliography}

\end{document}
    