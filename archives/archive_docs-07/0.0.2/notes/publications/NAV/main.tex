\documentclass[a4paper,12pt]{book}
\usepackage[T1]{fontenc}
\usepackage[utf8]{inputenc}
\usepackage{titlesec}
 
\usepackage{blindtext}
\usepackage{mathrsfs,amsmath}
\usepackage[utf8]{inputenc}
\usepackage{titlesec}
\usepackage{amsmath}
\usepackage{algorithm}
\usepackage[noend]{algpseudocode}
\usepackage[toc,page]{appendix}
\usepackage{graphicx}
\usepackage{lipsum}
%\documentclass{amsart}



\setcounter{secnumdepth}{4}

\titleformat{\paragraph}
{\normalfont\normalsize\bfseries}{\theparagraph}{1em}{}
\titlespacing*{\paragraph}
{0pt}{3.25ex plus 1ex minus .2ex}{1.5ex plus .2ex}




\title{NAV \\ System Design}
\author{eatondo}

\begin{document}  
\maketitle

\pagebreak


\tableofcontents{}

\pagebreak

NAV - Network of autonomous vehicles \\
\\- How can we improve human mobility ? \\
\indent - human capital \\
\indent - resources that humans need \\

Looking from the high level perspective of improving human mobility, through use of AV and providing a AV solution to the transportation problem, \\

I begin this work with the attempt to answer a simple question, How can one optimize human mobility through use of autonomous vehicles. Justification for the actions taken to attempt a solution to this problem is believed to best be available from a view from the top level, down to the visceral components of the design. 

\part{Mobility Layer}

\chapter{Introduction}

\end{document}
