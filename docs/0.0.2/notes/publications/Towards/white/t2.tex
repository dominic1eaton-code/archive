\documentclass[12pt]{book}

\renewcommand{\rmdefault}{ppl}
\renewcommand{\bfdefault}{b}
\usepackage[scaled]{helvet}
\renewcommand{\ttdefault}{pcr}
\linespread{1.05}
\normalfont % in case the EC fonts aren't available
\usepackage[T1]{fontenc}
\usepackage[utf8]{inputenc}
\usepackage{geometry}
\geometry{margin=1in}



\title{Toward Enhanced Mobility \\ \textit{white paper}}

\author{Dominic Eaton}
\date{February 2019}

\begin{document}
    \maketitle
    
    \tableofcontents
    
    \part{Preliminaries}
    
    \chapter{Introduction}
    
    Long have passed the days for which mobility was a right reserved for only those willing to bear the high price of traversing the land. As the advent of technological advances began mass proliferation, beginning with the simple wheel, all the until the current era of the vehicle, humanity has witnessed a fundamental shift ... . As it stands now, the possibility of yet another such sift seems imminent, and though it is not fully understood yet as to what this would mean for the current way of life of the modern human, ... . \\ 
    \indent To the ends of creating such a robust system, whose capabilities are not limited to the complexity or volatility of its environment, this work acts as a point of commencement on the study of the feasibility and nature of such systems. It is begun with the simple question of: How can we improve the mobility of humans ? It is from this point that it is concluded that autonomous vehicles endowed with intuitive capabilities for learning, adapting, and anticipating the needs of those to use it, will most likely provide the best answer to this question. The goal of this work is not to simply provide a solution to the "transportation problem" and only find an means for optimal human mobility in unpredictable environments, but to also take a step back and truly look at the concept of what it means to have a "good" mobility system that will endure the test of time, and at the most fundamental level, explore what would autonomous vehicle require in order to truly reach this goal. \\
    \indent Having these ideas in mind, ... . \\
    
        \section{The Scene}
        
        \indent Very interesting paper on a more or less "novel" deep neural architecture that seems to achieve good performance, while utilizing much less computational resources.
        I know that right now, you guys are focusing on simpler learning problems, which are mainly convex in nature, but perhaps you may be interested in deep learning or non-convex / linear learning problems at some point in the future.
        I can imagine the possibility of great potential in this new type of neural network, especially when combined with reinforcement learning techniques and predictive control. \\
    
        Perhaps such an architecture could increase the efficiency/efficacy/utility of a distributed learning system (such as the one recently proposed by google), where learning models exist on distributed devices, and furthermore could be used in a system of distributed learning agents who have limited computational capacity and/or communication capabilities.As you could probably imagine, such a system could include a network of autonomous vehicles, who are collectively trying to learn a series of (control/prediction) tasks. The autonomous vehicles only have a partial view or only have partial information about their environment, but together, the entire network of autonomous vehicles are able to create a very collectively rich/good representation or model of their overall environment. The autonomous vehicles (AVs) are able to communicate with one another through a vehicular ad-hoc network, which is essentially a distributed network, but there may not exist a centralized or even decentralized server or road side unit (RSU) for them to utilize. Therefore, in the worst case, all communication and agent interaction is purely distributed, and this presents a challenging setting for these group of autonomous vehicles to overcome.To accomplish the goal of coordinated learning and control, using only distributed interaction and communication, perhaps an similar architecture, as presented by Google and some other research groups (namely "DeepChain: Auditable and Privacy-Preserving Deep Learning with Blockchain-based Incentive") could be leveraged, and combined with (Secure) Federated Averaging or Secure Mean Field aggregation techniques, which will allow agents to build (predictive) model representations of their environment and other agents around them. These (predictive) model representations, constructed by each agent, could be the foundation that agents to begin taking actions, making decision, making and refining predictions, refining behavior, learning, etc..., with the hope of ultimately leading to emergent behavior arising from collective intelligence (the intelligence that is now being shared through all of the agents).So, a distributed learning mechanism, which is private by design (perhaps using differential privacy, secure multiparty computations, and blockchain/cryptoeconomic concepts, which works in conjunction with an incentive mechanism design system (for facilitating data sharing (as data is extremely important for learning algorithms currently), could begin to form the physical realization to accomplish the goal of creating a collective intelligent system (network) of autonomous vehicles who can act, predict, and learn optimally based only on their local perceptions, and past experiences. Beyond just the creation of such a system, one can begin to use this system for a wide variety of applications. An application that comes to mind is the crowding of autonomous vehicle to create stable communication network to facilitate (allow for) the creation and utilization of edge computing applications.Edge computing applications in a distributed vehicular ad-hoc network will most likely require stable, long lasting connections to be maintained by the vehicles. So vehicles will need to be able to drive very consistently, for significantly long periods of time, while maintaining short distances, all while dealing with a very complex, noisy, and erratic environment. A network of autonomous vehicles who have, at the very the least, the capabilities and flexibility to perceive, learn, predict, communicate, efficiently memorize, interact and share/interpret high level information may in fact provide the best solution toward creating such a system which can allow an edge network to be feasible and sustainable for vehicular ad-hoc networks consisting of AVs.12:08 PMTo go even further, edge computing applications are only the baseline architecture for providing a general setting to allow the building/production of various applications. Complex applications are of course , in some ways, boundless, but being more concrete, one could essentially build applications to offload computation on these distributed vehicle edge networks, or do telemetry/telematic environmental sensing applications, or create any crowdsourcing or crowdsensing application, where the crowds not only consist of vehicles, but the people who are using them, or even create a form of distributed storage, where vasts of amounts of data can be stored on a mobile (moving) network/platform of agents allowing for increase information processing rates at readily definable locations in the world.All of this is to say, that in general, from the concepts that I have just talked about, and the concepts that I spent the entirety of last semester presenting on, and talking to both of you about, I truly believe that a network of autonomous vehicles, that has flexibility, fine control, dexterity and sustainability is perhaps very possible with the system that I have been outlining (for a while now), and even though this idea is in its infancy, I wanted you guys to be aware of the general idea, and how I have been trying for some time now, to connect all of these ideas together, in order to get to the central point of attempting to answer this question:"What would it take to optimize the mobility capabilities of humanity ?"or at the very least: "What would it take to (significantly) improve the mobility capabilities of humanity ?"A question, that I first proposed last semester (during my last/2nd to last presentation) and have now begun the path/journey toward answering, with the (rough) system design that I have proposed to you guys.So I hope now that this puts into further context what it is that I have been doing and had been presenting on since last semester, and really clarifies the overall goals and objectives, and hopefully you guys understand why I have been asking you both certain questions about your projects and other things, as I really want to convey/delineate the fact that I think that the existence of such a system would be of monumental benefit to a great many people yes, but furthermore, I think that the realization of the goals of this project are (hopefully) within reach and are feasible to obtain.So I basically want to say that, all of these papers that I have been sending to you guys, and all of my presentations, and really of the research and information , has in fact (or at least is now) been in regards to, or been for the purpose of realizing this goal and answering the two questions that I have previously posted.So everything has been in some sense connected
        
        \section{Proposal}
    
    
    
    \chapter{Overview}
    
    \chapter{Background}
    
        \section{Distributed Systems}
        
        \section{Networking}
        
        \section{Controls}

            \subsection{DMPC}
            
            \subsection{Formation Control}
            
        \section{Game Theory}
        
        \section{Cryptography}
        
        \section{Cryptoeconomics}
        
        \section{Governance}
        
        \section{Machine Learning}
        
        \section{Artificial Intelligence}
        
        \section{Collective Intelligence}
        
        \section{Statistical Mechanics}
        
        \section{Chaos Theory}


    \part{System Overview}
    
    \chapter{The Challenges}
    
        \section{The Cognition Challenge}
        
        Learning, Adapting, Negotiation, dealing with Rationality and Irrationality ... .
        
        \section{The Communications Challenge}
        
        Agent Interaction, Inter-Agent regulation and control ... .
        
        \section{The Control Challenge}
        Self regulation, actuation, allostasis, ... .
        
        \section{The Security Challenge}
        
        \section{The Economics Challenge}
        
        \section{The Organization Challenge}
    
    \chapter{The Protocol Stack}
        
        \section{Physical Layer}
        
        \section{Communication Layer}
        
        \section{Information Layer}
        
        \section{Intelligence Layer}
        
            \subsection{Affectation}
            
            \subsection{Behavior}
            
            \subsection{Cognition}
        
        \section{Application Layer}
        
        \section{Management Layer}
    
    \chapter{Desiderata}
    
    
    
    \part{The Agent}
    
    \chapter{Simulation}
    
        \section{Representation}
        
        
    \chapter{Interaction}
    
    \chapter{Learning}
    
    \chapter{Prediction}
    
    \part{System Design}
    
\end{document}
